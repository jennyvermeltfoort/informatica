
\documentclass[a4paper,11pt]{article}
\usepackage[a4paper, total={5in, 10in}]{geometry}
\frenchspacing
\usepackage{microtype}
\usepackage{graphicx}
\usepackage{float}
\usepackage{placeins}
\graphicspath{ {./images/} }
\usepackage[pdfauthor={Jenny Vermeltfoort},pdftitle={REPRESENTEREN WETTEN BINNEN HET VAKGEBIED INFORMATICA DE
            WETENSCHAPPELIJKHEID VAN DE DISCIPLINE?.}]{hyperref}
\usepackage[colaction]{multicol}
\setlength\columnsep{30pt}

\usepackage[columnwise]{lineno}
\modulolinenumbers[10]

\usepackage[dutch]{babel}
\selectlanguage{dutch}

\usepackage{sectsty}% http://ctan.org/pkg/sectsty
\usepackage{titlecaps}% http://ctan.org/pkg/titlecaps

\usepackage{lmodern}
\fontfamily{lmdh}\selectfont

\usepackage[backend=biber,style=ieee]{biblatex}
\addbibresource{bib.bib}

\begin{document}

\title{REPRESENTEREN WETTEN BINNEN HET VAKGEBIED INFORMATICA DE WETENSCHAPPELIJKHEID VAN DE DISCIPLINE?}
\author{Jenny Vermeltfoort, 3787494}
\date{\today}
\maketitle

\sectionfont{\centering\MakeUppercase}
\linenumbers

\section*{INLEIDING}
Is het vakgebied informatica, ofwel computerwetenschappen, een wetenschappelijk discipline? Een discussie met
verschillende meningen en een vraag die vrijwel voor elke discipline in de academische wereld van toepassing is;
een
vraag waar oneindig over gefilosofeerd kan worden. Paul Graham, ontwikkelaar van Yahoo! Store, bevind dat
informatica
een paraplu term is van dingen die zowel wetenschappelijk of technisch kunnen
zijn\cite[p~29]{denningComputerScienceScience2005}.  In tegenstelling
vind
Kurt D. Krebsbach, professor wiskunde aan de Lawrence universiteit, dat de computerwetenschappen gaat over het
ontwikkelen van instrumenten en betoogt daarmee dat het geen wetenschap is\cite{krebsbachComputerScienceNot2015}.

Wat is inherent aan de wetenschap en wat maakt een vakgebied wetenschappelijk; is het simpelweg het gebruik van de
wetenschappelijke methode?

Binnen het kader van empirisch onderzoek is het vormen van wetmatigheden is een belangrijk onderdeel.
Dit
verslag zal een opinie vormen over wetmatigheden binnen het vakgebied informatica, specifiek de wetten van Moore en
Wirth, en betogen dat deze wetten de wetenschappelijkheid van het vakgebied informatica niet aantonen.
\section*{WET OF DOEL}
In 1965 voorspelde zakenman, engineer, en medeoprichter van Intel, Gordon Moore dat de trend waarmee de hoeveel
transistoren op een silicon jaarlijks exponentieel zou groeien; een voorspelling die tien jaar later werd aangepast
naar tweejaarlijks en die vanwege haar realistische karakter tot wet werd benoemd. Het bijzondere is dat deze wet
eenenvijftig jaar lang een correcte voorspelling is geweest en in 2016 tot een eind is
gekomen\cite[p~1]{leisersonTherePlentyRoom2020}.

Moore was ervan bewust dat zijn voorspelling een psychologisch effect had op fabrikanten; het werd een zelf
vervullende profetie\cite[p~55]{schallerMooreLawPresent1997}. In Moore’s eigen woorden:``it’s about vision, it’s about
what you’re allowed to
believe``\cite[p~59]{schallerMooreLawPresent1997}. Fabrikanten deden er alles aan om deze visie werkelijkheid te
maken;
zoals bijvoorbeeld het
ontstaan van IBM, een alliantie van Duitse Siemens en Japanse Toshiba. Het gaat hier dus om collectief handelen wat
ervoor heeft gezorgd dat Moore’s voorspelling werkelijkheid werd. En kan er nog steeds gesproken worden over een
wet
als deze door menselijk handelen tot werkelijkheid kan worden gebracht?

Francis Bacon zegt het volgende over axioma:``Indeed in the establischment of any true axiom, the negative instance
is
the more forcible of the two [affirmation and disavowment]``\cite[p.~20]{baconBaconNewOrganon}. De bewering over
axioma is ook van
toepassing
op Moore’s wet en toont aan dat het hier niet om een wetenschappelijke wet gaat. Want hoe kan het negatieve van
Moore’s wet worden bewezen als deze door menselijk handelen tot werkelijkheid gemaakt kan worden? De wet wordt
onwaar
wanneer de industrie de voorspelling niet langer als doel voor ogen neemt. Dus de voorspelling is een reflectie van
de
collectieve inspanningen en overtuigingen van de industrie. En alhoewel sociale of economische beperkingen
limiterende factoren kunnen zijn, is er geen factor buiten de menselijke controle waardoor de wet onwaar zou kunnen
zijn.

Zou de voorspelling waar zijn geworden wanneer er een andere vorm van groei zou zijn omschreven? De wet is tot
einde
komen vanwege fysieke beperkingen, een factor waar de mens geen controle over heeft. De voorspelling gaat echter
niet
over fysieke haalbaarheid, er wordt geen eindjaartal genoemd, maar het is een voorspelling over periodieke
exponentiele
groei. Een voorspelling niet gebaseerd op natuurlijke beperkingen maar op het menselijk collectief en innovatief
vermogen om exponentiele groei te bewerkstelligen.

Moore’s initiële voorspelling is beperkt in empirisch bewijs; het negatieve kan niet worden bewezen en het duidt
eerder
op een doel of visie dan op een wet.
\section*{CONCLUSIE}
Het voorgaande sectie laat zien dat Moore’s voorspelling geen wetenschappelijke wet is. Dezelfde logica is van
toepassing op Wirth’s wet, ook een voorspelling gerealiseerd door menselijk handelen en waar het negatieve niet
van
vastgesteld kan worden\cite{wirthPleaLeanSoftware1995}. Het is daarbij duidelijk dat Moore’s voorspelling geen
natuurlijk fenomeen omschrijft,
maar
wat is het dan wel? Een economisch of psychologisch fenomeen? Is het überhaupt een computerwetenschappelijk fenomeen?
Wat is
een
computerwetenschappelijk fenomeen? Het wordt duidelijk dat de wetenschappelijkheid van de discipline informatica niet
aan te
tonen is
met deze twee wetten.

% the bibliography
\printbibliography

\section*{REFLECTIE}
Ik heb eerst de vijf gegeven papers gelezen om een idee te krijgen over wat ik wil schrijven. Tijdens natuurlijk
notities gemaakt. Vervolgens heb ik de notities die relevant zijn aan de hoofdvraag in een kort plan van twee
pagina’s
samengevoegd. Op een natuurlijke wijze kwamen ideeën op welke ik heb gevormd tot argumentatie. Ik heb
duckduckgo’s
ChatGPT geprobeerd nadat ik mijn verslag had geschreven. Dit heeft alleen wat aanpassingen in zinstructeren
opgeleverd; de inhoud is exact hetzelfde gebleven. Ik vond het daarbij prettig om een soort bevestiging te krijgen
over
wat ik heb geschreven en het heeft een klein beetje onzekerheid weggenomen. Nu weet ik dat wat ik heb geschreven
in
ieder geval geen complete onzin is.

\end{document}