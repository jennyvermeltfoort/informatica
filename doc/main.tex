
\documentclass[10pt]{article}

\usepackage[utf8]{inputenc}
\usepackage{xcolor}

\parindent=0pt
\usepackage{hyperref}
\hypersetup{
      colorlinks = true,
      linkcolor = blue,
      linkbordercolor = {white},
}
\usepackage{fullpage}

\frenchspacing

\usepackage{microtype}

\usepackage[english,dutch]{babel}

\usepackage{listings}
% Er zijn talloze parameters ...
\lstdefinestyle{myStyle}{
      belowcaptionskip=1\baselineskip,
      breaklines=true,
      frame=none,
      numbers=left, numberstyle=\tiny, numberfirstline=false, breaklines=true,
      stepnumber=1, tabsize=8,
      basicstyle=\footnotesize\ttfamily,
      keywordstyle=\bfseries\color{blue!40!black},
      commentstyle=\itshape\color{green!40!black},
      identifierstyle=\color{black},
      backgroundcolor=\color{gray!10!white},
}

\title{Opdracht 1}
\author{Jenny Vermeltfoort}

\begin{document}

\selectlanguage{dutch}

\maketitle

\section{Uitleg}
De applicatie formatteert C- of C++-bronbestanden, verwijdert met // voorziene opmerkingen en herstructureert de
inspringing. Het controleert of het invoerbestand een geldige accoladebalans heeft. Daarnaast wordt een door de
gebruiker opgegeven combinatie van letters geteld binnen het invoerbestand. Getallen in het invoerbestand worden
gecontroleerd op Lychrel-nummers met behulp van het Lychrel-algoritme. Zie \hyperref[sec:eisen]{paragraaf
      \ref{sec:eisen}} voor de eisen.

 

\section{Eisen}
\label{sec:eisen}
De applicatie voldoet aan de volgende eisen:
\begin{enumerate}
      \item De applicatie neemt een ingangsprogramma en produceerd een uitgangsprogramma. Het uitgangsprogramma is een
            getransformeerd ingangsprogramma.
      \item De applicatie verwijderd al het "//" commentaar van het ingangsprogramma.
      \item De applicatie formateerd het ingangsprogramma zodat de inspring per regel in het geheel exact is. Met een
            argument wordt aangegeven hoeveel spaties \'e\'en insprong is.
      \item De applicatie controleert of het aantal open accolades gelijk is aan het aantal gesloten accolades.
      \item De applicatie berekend hoevaak een met een argument ingegeven combinatie van letters voorkomt in het
            ingangsprogramma. Inclusief commentaar, hoofdlettergebruik wordt genegeerd, "abc" is dus gelijk aan
            "ABC".
      \item De applicatie bepaald voor elk getal groter dan nul in het ingangsprogramma of het een Lychrel-getal is.
            Verificatie
            wordt tot INT\_MAX uitgevoerd. In het CLI-interface wordt het Lychrel-getal afgedrukt en het aantal
            iteraties
            noodzakelijk om het palinroom te behalen. Bij INT\_MAX iteraties of meer wordt gemeld dat de applicatie het
            niet weet voor dit
            getal. Getallen binnen commentaar worden genegeerd. Enkel invoergetallen tussen 1 en INT\_MAX voldoen.
      \item De applicatie verwacht met een argument de locatie van het ingangsprogramma. Wanneer het ingangsprogramma
            niet bestaat stopt het programma.
      \item De applicatie mag enkel het ingangsprogramma lezen.
      \item De applicatie verwacht een foutloos C++ programma. I.e. het ingangsprogramma moet te compileren zijn met
            een
            g++ compiler.
      \item De applicatie controleert niet of het ingangsprogramma foutloos is.
      \item De applicatie verwacht met een argument de locatie van het uitgangsprogramma. Wanneer het uitgangsprogramma
            niet bestaat wordt het aangemaakt, wanneer het wel bestaat wordt het overschreven.
      \item De applicatie print aan het eind van het programma het volgende in het CLI-interface:
            \begin{enumerate}
                  \item Het aantal optreden van de ingevoerde combinatie van letters.
                  \item Het resultaat van het accolade verificatie algoritme.
            \end{enumerate}
\end{enumerate}

\section{Gebruik}
Het programma kan met de volgende call aangeroepen worden:
\lstinputlisting[caption=usage.txt,
label={lst:listing-usage},
language=C++,
style=myStyle]{./logs/usage.txt}

Een voorbeeld volgt: snipped \ref{lst:listing-input} toont een voorbeeld input file en snippeds \ref{lst:listing-stdout} en \ref{lst:listing-output} laten het resultaat zien van de applicatie.

\lstinputlisting[caption=input.txt,
label={lst:listing-input},
language=C++,
style=myStyle]{./logs/test.txt}
\lstinputlisting[caption=stdout.txt,
label={lst:listing-stdout},
language=bash,
style=myStyle]{./logs/stdout.log}
\lstinputlisting[caption=output.txt,
label={lst:listing-output},
language=C++,
style=myStyle]{./logs/output.txt}

\section{Vragen}
\textit{Stel dat je een door het programma netjes afgedrukte file F opnieuw met het programma netjes af laat drukken. Is het resultaat dan altijd precies hetzelfde als F?}
\newline Het format (fs\_format) process verwijderd commentaar en herproduceerd de indenting van de file. 
Aangezien het commentaar in F is verwijderd zal deze logica niets doen bij opeenvolgende iteraties.
De applicatie is geschreven met het idee dat de indenting van de file word geherproduceerd, 
met de nadruk op \textit{her}. De logica zal daarmee exact dezelfde indenting produceren voor elke 
regel onafhankelijk van het aantal iteraties waarmee F door de applicatie wordt gehaald.
Tenzij een bug ervoor zorgt dat er meer indents geproduceerd of geconsumeerd worden zal 
F dus altijd gelijk blijven. Het is ook mogelijk dat de gebruiker een andere tabsize gebruikt, 
wat voor variatie kan zorgen.
\\ \hfill \\
\textit{Voor grondtal 2 bestaan Lychrel-getallen. Geef er één, en bewijs de eigenschap.}
\newline 


\section{Tijd}
Zie \ref{tab:time} voor de tijdsverantwoording.
\\ \hfill \\
\begin{tabular}{ l c r }
      \label{tab:time}
      Item & Uur & Datum \\ \hline
      Eisen verwerkt & 1 & 22-09-2024 \\
      Plan bedacht & 1 & 22-09-2024 \\
      Argument parser & 1 & 26-09-2024 \\
      File formatter & 2 & 26-09-2024 \\
      Letter counter & 0.5 & 26-09-2024 \\
      Lychrel algoritme & 2 & 26-09-2024 \\
      Afwerking & 1 & 27-09-2024 \\
    \end{tabular}

\section{Code}\label{sec:code}
\lstinputlisting[caption=main.cpp,
      label={lst:listing-cpp},
      language=C++,
      style=myStyle]{../src/main.cc}



\end{document}