
\documentclass[10pt]{article}

\usepackage[utf8]{inputenc}
\usepackage{xcolor}

\parindent=0pt
\usepackage{hyperref}
\hypersetup{
  colorlinks = true,
  linkcolor = blue,
  linkbordercolor = {white},
}
\usepackage{fullpage}

\frenchspacing

\usepackage{microtype}

\usepackage[english,dutch]{babel}

\usepackage{listings}
% Er zijn talloze parameters ...
\lstdefinestyle{myStyle}{
  belowcaptionskip=1\baselineskip,
  breaklines=true,
  frame=none,
  numbers=none,
  basicstyle=\footnotesize\ttfamily,
  keywordstyle=\bfseries\color{blue!40!black},
  commentstyle=\itshape\color{green!40!black},
  identifierstyle=\color{black},
  backgroundcolor=\color{gray!10!white},
}

\title{Opdracht 1}
\author{Jenny Vermeltfoort}

\begin{document}

\selectlanguage{dutch}

\maketitle

\section{Uitleg}
De applicatie is een vragenformulier welke valideerd of een gebruiker geschikt is voor een universitaire studie. Via
een CLI-interface worden er vragen gesteld en er wordt verwacht dat de gebruiker input levert conforme de gevraagde
format per vraag. De source code is te vinden in paragraaf \hyperref[sec:code]{paragraaf \ref{sec:code}}. De source
code is getest met de test beschreven in \hyperref[sec:test]{paragraaf \ref{sec:test}}, zie subparagraaf
\hyperref[sec:resultaat]{paragraaf \ref{sec:resultaat}} voor de
resultaten.
\section{Eisen}
De applicatie voldoet aan de volgende eisen:
\begin{enumerate}
  \item De gebruiker wordt gevraagd om zijn of haar geboortejaar.
  \item De gebruiker mag niet jonger zijn dan 10 jaar, dit wordt getest bij elke prompt dat gerelateerd is aan de
        geboortedatum.
  \item Wanneer er een geboortejaar jonger dan nu wordt gegeven, wordt input gekenmerkt als niet valide.
  \item De gebruiker wordt gevraagd om zijn of haar geboortemaand.
  \item Wanneer er een geboortemaand wordt gegeven buiten de criteria \{1..12\} dan wordt de input gekenmerkt als niet
        valide.
  \item De gebruiker wordt gevraagd om zijn of haar geboortedag numeriek.
  \item Wanneer er een geboortedag wordt gegeven buiten de criteria \{1..31\} dan wordt de input gekenmerkt als niet
        valide.
  \item Wanneer er een geboortedag wordt gegeven buiten het criteria van de maand, wordt het verschil opgesomd bij de
        geboortemaand. Bijvoorbeeld wanneer de gebruiker in geeft: \"1994 2 30\" (1994/2/30) dan wordt de datum
        (1994/3/2).
  \item Wanneer de gebruiker jonger is dan 30 jaar worden de hierna komende vragen informeel gesteld.
  \item De gebruiker wordt gevraagd om zijn of haar geboortedag text (ma,di,wo,do,vr,za,zo).
  \item Wanneer de geboortedag in text formaat niet overeenkomt met de dag van de ingegeven geboortedatum worden de
        gegevens van de gebruiker verwijderd en stopt het prompt.
  \item De gebruiker wordt gevraagd om een wiskunde probleem op te lossen. Er wordt gevraagd om een
        som van twee breuken, waar de tellers en noemers van de breuken tussen de 1 en 19 liggen.\
  \item Het antwoord wordt gedeeld in drie prompts: teller, noemer, en het antwoord met twee decimalen achter de komma.
  \item Bij elk prompt wordt er gevalideerd of het antwoord op de wiskundevraag klopt.
  \item Het decimalen antwoord op de wiskundevraag heeft een foutmarge van 0.1.
  \item Wanneer er een foutief antwoord wordt geleverd wordt er een literatuur vraag gesteld. Er wordt een
        meerkeuzevraag gesteld met 4 keuzes, antwoord (b) is altijd correct.
  \item De keuzes van de meerkeuzevraag worden gelabeld met (a,b,c,d).
  \item Het formaat van de input correspondeert met de meerkeuzevragen: (a,b,c,d).
  \item Wanneer de input niet correspondeert met het formaat wordt de input als niet valide gekenmerkt.
  \item Wanneer er een foutief antwoord wordt geleverd is de gebruiker niet geschikt voor een universitaire studie en
        eindigt het prompt.
  \item Wanneer er een correct antwoord wordt geleverd op de wiskunde vraag is de gebruiker geschikt voor een exacte
        universitaire studie en eindigt het prompt.
  \item Wanneer er een correct antwoord wordt geleverd op de literatuur vraag is de gebruiker geschikt voor een beta
        universitaire studie en eindigt het prompt.
\end{enumerate}

\section{Tijd}
Het heeft ongeveer 10 uur gekost.

\section{Code}\label{sec:code}
\lstinputlisting[caption=main.cpp,
  label={lst:listing-cpp},
  language=C++,
  style=myStyle]{../src/main.cpp}

\section{Test}\label{sec:test}
\lstinputlisting[caption=test.sh,
  label={lst:test},
  language=sh,
  style=myStyle]{../test.sh}

\subsection{Resultaat}\label{sec:resultaat}
\lstinputlisting[caption=test.sh,
  label={lst:test-result},
  language=sh,
  style=myStyle]{./results.log}

\end{document}
